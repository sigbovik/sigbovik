%%%%%%%%%%%%%%%%%%%%%%%%%%%%%%%%%%%%%%%%%%%%%%%%%
%%%
%%% SIGBOVIK 2020 TRIPLE BLIND REVIEW TEMPLATE
%%%
%%% Instructions:
%%%
%%% 1. Edit the author, rating, and confidence
%%%    fields.
%%% 2. Enter your review after the \maketitle
%%%    command
%%%
%%%%%%%%%%%%%%%%%%%%%%%%%%%%%%%%%%%%%%%%%%%%%%%%%
\documentclass[12pt]{sigbovik-review}

%%%% Edit the following three lines.
%% To ensure the integrity of the triple-blind review
%% process, the contents of the \author field should not
%% reveal your identity.
\author{Skirt Steak}
\rating{100\%}
\confidence{Kim Possibly Confident}

%%%% The Proceedings Chair will fill in the following
%%%% two fields.
\papernum{9}
\papertitle{Can a paper be written entirely in the title? 1. Introduction: The title pretty much says it all. 2. Evaluation: It gets the job done. However, the title is a little long. 3. Conclusion: Yes.}

\begin{document}

\maketitle

%%% Replace the following text with your review.
While the immediate academic implications of this paper are deep and far-reaching, a more thorough analysis reveals that it also has carefully hidden, previously undiscovered lore on the hit American television series Cory in the House, featuring Kyle Massey and Jason Dolley.

The first tip-off is found even before reaching the paper itself in the author's pseudonym, which is a pseudonym for "Cory Baxter". This is a subtle nod to Cory Baxter, the main character of Cory in the House.

In the second paragraph, we find the real hidden jewel of lore. By taking the first letter of every sentence in this paragraph and placing them in reverse order, a working URL is revealed. Each time the URL is opened, it has a 50\% to redirect to either of two pages: Webkinz.com, and the Wikipedia page for Jeff Bezos. It is a little known fact that Cory's favorite game is Webkinz. It follows that Cory's favorite multibillionaire American internet and aerospace entrepreneur, media proprietor, and investor is Jeff Bezos.

An extra nugget of lore is planted in the final paragraph, in the last sentence, in the last letter. By examining the lower edge of the bottom right serif with an electron microscope, you will find Cory himself. This has huge implications for the hit American television series Cory in the House, featuring Kyle Massey and Dan Schneider. The only explanation for this observation is that Cory is only slightly taller than a hydrogen atom, and the entire series had to be recorded by shrinking down the rest of the cast and set to an atomic size.

The effort required to make these details subtle yet discernible suggests that the author intended to communicate them secretly to some as-of-yet unidentified group or individual. But the public deserves to hear the truth.




\end{document}
