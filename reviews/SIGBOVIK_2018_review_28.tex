\documentclass[12pt]{article}
\usepackage{graphicx}
\usepackage{multirow}
\usepackage{amsmath}
\usepackage{fullpage}
\usepackage{times}
\setlength\parindent{0pt}
\setlength\parskip{12pt}

\begin{document}

{\sffamily
\begin{tabular}{ll}
\multirow{3}{*}{\includegraphics[width=1in]{ach.png}}\\
& \Large{\em CONFIDENTIAL COMMITTEE MATERIALS} \\
&\\
& \textbf{\Huge{SIGBOVIK 2018 Paper Review}} \\
&\\
& \LARGE{Paper 28: Substitute Teacher Networks} \\
&\\
\hline
\end{tabular}}
\vspace{2em}
\thispagestyle{empty}

{\large\bf
\begin{tabular}{l}
Ben Blum, Light Cone Sedentarian \\
Rating: Defer \\
Confidence: Righteous \\
\end{tabular}}
\vspace{1em}

As to the authors' perspective I no doubt have already written, but to my own have yet to write,
in my review of (Albanie et al., 2019),
the use of forward citations to one's own future work is an irresponsible act
which degrades the fabric of academic space-time.
I cannot condone this practice %, which I definitely have never myself done,
and recommend the paper's publication be deferred until 2020.

\vspace{3em}

{\large\bf
\begin{tabular}{l}
Reviewer Two, Association for Confectionery Heresy \\
Rating: Accept \\
Confidence: Just here for the cake \\
\end{tabular}}
\vspace{1em}

The paper makes important advances in the area of research paper structure.
Specifically, the unrelated work section helps ground the reader by providing a sense of the scope of the work,
the use of inspirational quotes is inspiring,
and the use of the intermission section (first proposed in SIGBOVIK Track L by [R. Two, 2015])
helps to keep the reader's attention.
I look forward to future work from these authors on incorporating those concepts into their teaching network itself
to confer the same benefit upon students.


\end{document}
