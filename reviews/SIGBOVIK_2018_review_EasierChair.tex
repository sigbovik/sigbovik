\documentclass[12pt]{article}
\usepackage{graphicx}
\usepackage{multirow}
\usepackage{amsmath}
\usepackage{fullpage}
\usepackage{times}
\setlength\parindent{0pt}
\setlength\parskip{12pt}

\begin{document}

{\sffamily
\begin{tabular}{ll}
\multirow{3}{*}{\includegraphics[width=1in]{ach.png}}\\
& \Large{\em CONFIDENTIAL COMMITTEE MATERIALS} \\
&\\
& \textbf{\Huge{SIGBOVIK 2018 Paper Review}} \\
&\\
& \LARGE{Paper EasierChair: A Survey of} \\[0.25em]
& \LARGE{Hardware Multithreading} \\
&\\
\hline
\end{tabular}}
\vspace{2em}
\thispagestyle{empty}

{\large\bf
\begin{tabular}{l}
Jim McCann (CMU Textiles Lab)\\
Rating: 1.5 \\
Confidence: High \\
\end{tabular}}
\vspace{1em}

While hardware multithreading is something that is relevant to both textiles construction and computer science, and I am a strong advocate for more textiles-cs crossover work appearing in important venues like SIGBOVIK, I believe this paper needs to be extended.

One could discuss, for example, a modern Jacquard loom, which uses a {\em warp}\footnote{Just like in GPU computation, come to think of it.} of parallel threads that can be independently selected per-instruction.
%Such a masking primitive is very useful in constructing woven cloth, as it allows alternating warps to be ``picked'' by subsequent instructions.

Or, instead, take the odd case of a mechanical knitting machine, which, rather than operating on many threads with a single needle, operates on a single thread with many needles.
Indeed, given the way knitting machines are constructed, many dependent operations can be in flight on a single thread at the same time,
with the machine handling tension forwarding and global order resolution through careful construction.

Even in the consumer machine-sewing realm, it is common to see two-way multithreading:
most machines use a ``top thread'' which passes through the needle and a ``bottom thread'' that comes from a smaller spool called a bobbin.
More complicated machines called {\em sergers} use five- or even six-way multithreading in order to simultaneously stitch a seam and wrap thread around it to prevent fraying.
Of course, one might argue that because all the threads are just being interleaved by the hardware at an instruction level, this is more akin to ``hyperthreading'', but it seems like a debate worth having.

%Also, there several puns that I believe fall below the terrible standard required for publication.

Also, you should definitely cite some of my papers.

\end{document}
